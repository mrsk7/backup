\documentclass[a4paper,11pt]{scrartcl}
\usepackage[english,greek]{babel}
\usepackage[utf8x]{inputenc}
\usepackage{amsmath}
\usepackage{mathtools}
\DeclarePairedDelimiter{\abs}{\lvert}{\rvert}
% define the title
\author{Μιχάλης Ροζής - 03109704}
\title{Σχεδίαση Αναλογικών Μικροηλεκτρονικών Κυκλωμάτων}
\date{}
\subtitle{Δεύτερη Σειρά Ασκήσεων}
\begin{document}
\maketitle
\section*{Άσκηση 3.1}
Θεωρούμε τεχνολογία $0.18\mu{m}$.
Το ρεύμα $I_2$ δημιουργείται απο την τάση 1.8V η οποία εφαρμόζεται στην κατα σειρά αντίσταση $R_{ds2} + 5k\Omega + R_{ds1}$.
\begin{gather}
		R_{ds1} = \frac{L}{\lambda L_1I_2} =\frac{0,25\mu{m}}{\lambda L_1I_2} = \frac{3,125}{I_2} \\
		R_{ds2} = \frac{L}{\lambda L_2I_2} =\frac{0,25\mu{m}}{\lambda L_2I_2} = \frac{3,125}{I_2}
\end{gather}
Άρα $R_{t} = R_{ds1} + R_{ds2} + 5k\Omega = \frac{3,125}{I_2} + \frac{3,125}{I_2} + 5\times 10^3 = \frac{6,25}{I_2} + 5\times10^{-3} \Omega $.
Τελικά $I_2 = \frac{1,8}{R_t} = \frac{1,8V}{\frac{6,25}{I_2}+5\times10^{-3} \Omega} \Rightarrow 1,8 = 6,25 + 5\times10^{-3}I_2 \Rightarrow I_2 = 890\mu A$


\end{document}
